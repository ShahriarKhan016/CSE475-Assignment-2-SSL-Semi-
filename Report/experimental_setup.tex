% ============================================================================
% EXPERIMENTAL SETUP
% ============================================================================
\section{Experimental Setup}
\label{sec:experimental_setup}

This section details the computational environment, training configurations, and experimental protocols used in this project.

\subsection{Computational Environment}

All experiments were conducted using cloud-based GPU computing resources:

\begin{table}[H]
\centering
\caption{Computational Environment}
\label{tab:compute_env}
\begin{tabular}{@{}ll@{}}
\toprule
\textbf{Resource} & \textbf{Specification} \\
\midrule
Platform & Kaggle Notebooks \\
GPU & NVIDIA Tesla P100 (16 GB) \\
CPU & Intel Xeon (4 cores) \\
RAM & 16 GB \\
Framework & PyTorch 2.0+ \\
YOLO Implementation & Ultralytics \\
\bottomrule
\end{tabular}
\end{table}

\subsection{Experiment Groups}

The experiments were organized into three main groups corresponding to the project objectives:

\subsubsection{Experiment Group 1: Supervised Learning Baseline}

\begin{table}[H]
\centering
\caption{Baseline Experiment Configuration}
\label{tab:baseline_config}
\begin{tabular}{@{}llll@{}}
\toprule
\textbf{Model} & \textbf{Epochs} & \textbf{Batch Size} & \textbf{Learning Rate} \\
\midrule
YOLOv10n & 100 & 16 & 0.01 \\
YOLOv11n & 100 & 16 & 0.01 \\
YOLOv12n & 100 & 16 & 0.01 \\
\bottomrule
\end{tabular}
\end{table}

\textbf{Data Augmentation} (applied during training):
\begin{itemize}[noitemsep]
    \item Mosaic augmentation
    \item Random HSV shifts
    \item Random horizontal/vertical flips
    \item Random scaling and translation
\end{itemize}

\subsubsection{Experiment Group 2: Semi-Supervised Object Detection}

\begin{table}[H]
\centering
\caption{Semi-Supervised Experiment Configuration}
\label{tab:ssod_config}
\begin{tabular}{@{}ll@{}}
\toprule
\textbf{Parameter} & \textbf{Value} \\
\midrule
\multicolumn{2}{c}{\textit{Teacher Model}} \\
\midrule
Base Architecture & YOLOv12 \\
Training Data & Labeled only (20\%) \\
Epochs & 100 \\
\midrule
\multicolumn{2}{c}{\textit{Pseudo-Label Generation}} \\
\midrule
Confidence Threshold ($\tau$) & 0.70 \\
NMS IoU Threshold & 0.45 \\
\midrule
\multicolumn{2}{c}{\textit{Student Model}} \\
\midrule
Base Architecture & YOLOv12 \\
Training Data & Labeled + Pseudo-labeled \\
Epochs & 100 \\
\bottomrule
\end{tabular}
\end{table}

\subsubsection{Experiment Group 3: Self-Supervised Learning}

\paragraph{SimCLR Pretraining Configuration}

\begin{table}[H]
\centering
\caption{SimCLR Pretraining Configuration}
\label{tab:simclr_pretrain}
\begin{tabular}{@{}ll@{}}
\toprule
\textbf{Parameter} & \textbf{Value} \\
\midrule
Backbone & ResNet-18 \\
Projection Dimension & 128 \\
Temperature ($\tau$) & 0.07 \\
Batch Size & 32 \\
Epochs & 100 \\
Optimizer & Adam \\
Learning Rate & 0.001 \\
LR Schedule & Cosine annealing \\
Weight Decay & 1e-4 \\
\bottomrule
\end{tabular}
\end{table}

\paragraph{SimCLR Fine-tuning Configuration}

\begin{table}[H]
\centering
\caption{SimCLR Fine-tuning Configuration}
\label{tab:simclr_finetune}
\begin{tabular}{@{}lll@{}}
\toprule
\textbf{Parameter} & \textbf{Linear Eval} & \textbf{Full Fine-tune} \\
\midrule
Encoder & Frozen & Trainable \\
Classifier & 512 $\rightarrow$ 3 & 512 $\rightarrow$ 3 \\
Epochs & 50 & 50 \\
Batch Size & 32 & 32 \\
Learning Rate & 0.01 & 0.001 \\
Optimizer & SGD & Adam \\
\bottomrule
\end{tabular}
\end{table}

\paragraph{DINOv3 Configuration}

\begin{table}[H]
\centering
\caption{DINOv3 Feature Extraction and Classification}
\label{tab:dino_config}
\begin{tabular}{@{}ll@{}}
\toprule
\textbf{Parameter} & \textbf{Value} \\
\midrule
\multicolumn{2}{c}{\textit{Feature Extraction}} \\
\midrule
Pretrained Model & DINOv3 ViT-B/16 \\
Feature Dimension & 768 \\
Source & Facebook AI Research \\
\midrule
\multicolumn{2}{c}{\textit{MLP Classifier}} \\
\midrule
Architecture & 768 $\rightarrow$ 256 $\rightarrow$ 128 $\rightarrow$ 3 \\
Activation & ReLU \\
Dropout & 0.3 \\
Epochs & 100 \\
Learning Rate & 0.001 \\
Optimizer & Adam \\
\midrule
\multicolumn{2}{c}{\textit{YOLO Integration}} \\
\midrule
Base Model & YOLOv12 \\
Training Epochs & 20 \\
Batch Size & 16 \\
\bottomrule
\end{tabular}
\end{table}

\subsection{Training Procedures}

\subsubsection{Loss Functions}

\paragraph{Object Detection Loss (YOLO)}
The YOLO models use a composite loss function:
\begin{equation}
\mathcal{L}_{\text{YOLO}} = \lambda_{\text{box}} \mathcal{L}_{\text{box}} + \lambda_{\text{cls}} \mathcal{L}_{\text{cls}} + \lambda_{\text{obj}} \mathcal{L}_{\text{obj}}
\end{equation}
where $\mathcal{L}_{\text{box}}$ is the CIoU loss for bounding box regression, $\mathcal{L}_{\text{cls}}$ is the classification loss, and $\mathcal{L}_{\text{obj}}$ is the objectness loss.

\paragraph{Contrastive Loss (SimCLR)}
NT-Xent loss as described in Section \ref{sec:methodology}.

\paragraph{Classification Loss (DINOv3 MLP)}
Cross-entropy loss for multi-class classification:
\begin{equation}
\mathcal{L}_{\text{CE}} = -\sum_{c=1}^{C} y_c \log(\hat{y}_c)
\end{equation}

\subsubsection{Early Stopping and Checkpointing}

All experiments employed the following strategies:
\begin{itemize}
    \item Model checkpointing based on best validation mAP
    \item Early stopping with patience of 20 epochs
    \item Learning rate reduction on plateau
\end{itemize}

\subsection{Evaluation Protocol}

\begin{enumerate}
    \item Train models according to specified configurations
    \item Evaluate on validation set for hyperparameter selection
    \item Report final metrics on held-out test set
    \item Generate visualizations including:
    \begin{itemize}
        \item Training curves (loss, mAP evolution)
        \item Confusion matrices
        \item Precision-Recall curves
        \item Sample detection outputs
    \end{itemize}
\end{enumerate}
